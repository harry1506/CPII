\documentclass[a4paper]{scrartcl}

\usepackage[latin1]{inputenc}
\usepackage[T1]{fontenc}
\usepackage{lmodern}
\usepackage[ngerman]{babel}
\usepackage{amsmath}
\usepackage{color} %red, green, blue, yellow, cyan, magenta, black, white
\definecolor{mylightgray}{rgb}{0.95,0.95,0.95}
\usepackage{listings}

\lstset{ language=Matlab, 
  numbers=left,			
%  numberstyle=\footnotesize, 
  numbersep = 0.5cm,   	% Abstand zwischen Nummer und Code
  framexleftmargin=1cm, % Ramen links weiter ziehen
  xleftmargin=1cm,	  	% Ausrichtung korrigieren
  basicstyle=\footnotesize,
  showstringspaces=false,
  frame=lines,%lines, %leftline
  backgroundcolor=\color{mylightgray},
%  linewidth=0.95\linewidth,
  breaklines=true,
  morekeywords={Smetric},
 }

\title{Computational Physics II - �bung 1}
\author{Harald Krause, Oliver Hatt}
\date{4. Oktober 2013}

\begin{document}
\maketitle

\section*{Aufgabe 1.1:}
Es sollte eine reelle, symmetrische Matrix mit Hilfe der Jacobi Methode in Diagonalform �berf�hrt werden. Alle Bezeichnungen werden aus dem Script Computational Physics II, Ulli Wolff, WS 2013/14, Modul 22.1 Masterstudiengang Physik entnommen.

Hierzu werden sukzessive orthogonale Drehungen $P_{p,q}$ in jeder m�glichen p-q-Ebene durchgef�hrt. Dies geschieht derart, dass bei jeder Drehung immer in zwei Nicht-Diagonalelementen
\\\\
$$a'_{pq}=(c^2-s^2)a_{pq}+s c (a_{pp}-a_{qq})$$

\noindent Null entsteht.

\subsection*{a)}

F�r die Implementierung wurden wie vorgeschlagen die Formeln (1.27) bis (1.31) mit den auf Seite 9 gegebenen Definitionen von c, s und $\tau$ verwendet. Die dazu n�tigen Umformungsschritte werden in Teilaufgabe c) erl�utert.\\\\
Au�erdem wurden in der Matrix nur die Zeilen mit Hilfe von $i$ durchlaufen und so nur die untere Dreiecksmatrix berechnet (da $p<q$). Da in unserem Fall A symmetrisch ist und wir orthogonale Transformationen durchf�hren, ergeben sich die selben Elemente f�r die obere Halbmatrix und es spart Rechenaufwand diese Eintr�ge direkt zu �bernehmen.\\\\
Es l�sst sich nun beobachten wie pro Iteration die Halbmatrizen in s�mtlichen Elementen gegen Null streben. Das Programm bricht die Berechnung ab, sobald alle Nichtdiagonalelemente auf Maschinengenauigkeit Null geworden sind.\\
Die Funktion \texttt{doPpq} erzeugt f�r jeden Schritt (jedes Paar p-q) eine immer mehr diagonalisierte Matrix, nach dem oben beschriebenen Verfahren, au�erdem enth�lt sie noch einen Z�hler f�r die ausgef�hrten Multiplikationen.
F�r Aufgabenteil b) wird auch die Drehmatrix $P$ f�r die aktuelle $p-q$-Ebene zur�ckgegeben:\\\\
\lstinputlisting{doPpq.m}

\subsection*{b)}
F�r die Berechnung der Eigenvektoren der Matrix D wurden mit der Vorschrift (1.18) parallel die jeweiligen einzelnen Drehmatrizen $P_1 , P_2 , ...$ berechnet und aufeinander angewandt $(P = P_1 \cdot P_2 \cdot ...)$ .\\
Durch anschlie�ende Multiplikation von $V$ und $V^T$ ergibt sich die Einheitsmatrix und V ist daher orthogonal nach Definition.\\\\
Im folgenden Matlab-Code ist die Funktion myEig dargestellt. Sie berechnet die Eigenwerte und die Drehmatrix nach einer vorgebenen Zahl an maximalen Iterationen (bzw. 1000 default). Weiterhin gibt sie die Zahl Zahl der Iterationen zur�ck f�r die Anzahl an Iterationen, bis sich die Nicht Diagonalelemente auf Maschinengenauigkeit nicht mehr von Null unterscheiden:\\
\lstinputlisting{myEig.m}

\subsection*{c)}
Im Folgenden wird auf die Umformungen von (1.19), (1.20) auf (1.27), (1.28) eingegangen.\\\\

Aus der Multiplikation von $V$ mit $P$ erh�lt man (1.19) und (1.20):
\begin{align}
v_{i,p}' &= c\, v_{i,p} - s\,v_{i,q} \\
\quad 
v_{i,q}' &= s\, v_{i,p} + c\,v_{i,q}
\end{align}
(1) soll nach (1.27) gleich sein zu:
\begin{align}
v_{i,p}' &= v_{i,p} - s\,(v_{i,q} + \tau \, v_{i,p})\\
&= (1 - s \, \tau)\, v_{i,p} - s \, v_{i,q}
\end{align}
Der Vergleich von (1) mit (4) zeigt, dass lediglich der Koeffizient von $v_{i,p}$ u
umgeformt werden muss:
\begin{align}
1 - s \, \tau & = \frac{1+c}{1+c} - \frac{s^2}{1+c}
&\mbox{mit } \tau := \frac{s}{1 + c} \\
&= \frac{1 - s^2 + c}{1+c}\\
&= \frac{c \, (c+1)}{c+1}
&\mbox{mit } c^2 = 1 - s^2\\
&= c
\end{align}
F�r die Umformung von (1.20) zu (1.28) kann nun $c$ in (2) eingesetzt werden und liefert die zweite Gleichung:
\begin{align}
v_{i,q}' &= v_{i,q} + s\,(v_{i,p} - \tau \, v_{i,q})
\end{align}
\subsection*{d)}
Zur bestimmt der ben�tigten Multiplikationen wurde ein Z�hler in das Programm implementiert.\\
F�r die Auswerung der Flops in Abh�ngigkeit der Matrixdimension, wird eine Schleife ausgef�hrt, bei der die Dimension von $n=4$ bis auf $n=50$ erh�ht wird und jedes Mal die Multiplikationen gez�hlt werden. Im logarithmischen Plot kann ein konstanter Exponent verifiziert werden und der Exponent durch einen Polynomialfit bestimmt werden.\\
Im Folgenden nun zum Abschluss das Script f�r die Berechnungen wurde eine 50 x 50 Zufallsmatrix erzeugt und immer der Dimension entsprechend ein Teil von ihr als A deklariert:\\
\lstinputlisting{UE01_01a.m}
\end{document}

